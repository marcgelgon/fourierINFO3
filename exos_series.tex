


\exercice
Soient $(z_1,z_2)\in \C^2$. Montrer que $arg(z_1z_2)=arg(z_1)+arg(z_2)$

\exercice 
Démontrer les simplifications dans l'expression des coefficients de Fourier, dans le cas des fonctions paires et impaires.

\exercice 
Montrer l'expression des coefficients complexes (\ref{version_complexe} du poly) à partir de la définition de la série de Fourier (expression \ref{defseries}). 

\exercice
Montrer les expressions des coefficients de Fourier d'une dérivée (\ref{eqderivee} et trois suivantes).

\exercice
Développer la fonction $f$, de période $2\pi$, en série de Fourier, en utilisant les expressions réelles, puis les expressions complexes de la série et de ses coefficients.

\begin{equation*}
f(t)=
\begin{cases}
-1 & \text{si}\qquad t\in ]-\pi,0] \\
1 & \text{si}\qquad t\in ]0,\pi]
\end{cases}
\end{equation*} 

\exercice
Développer la fonction $f$, de période 4, en série de Fourier.

\begin{equation*}
f(t)=
\begin{cases}
 0 & \text{si}\qquad t\in ]-2,-1] \\
 k & \text{si}\qquad t\in ]-1,1] \\
 0 & \text{si}\qquad t\in ]1,2] 
\end{cases}
\end{equation*} 


\exercice
Développer la fonction $f$, de période T, en série de Fourier.

\begin{equation*}
f(t)=
\begin{cases}
 0 & \text{si}\qquad t\in ]-\frac{T}{2},0] \\
 sin(\omega t) & \text{si}\qquad t\in ]0,\frac{T}{2}]
\end{cases}
\end{equation*} 
où $T=\frac{2\pi}{\omega}$ établit le lien entre $T$ et $\omega$

\exercice
Développer la fonction $f$, de période $2\pi$, en série de Fourier. Elle est définie dans $]-\pi,\pi]$ par :
\begin{equation*}
f(t)= t+\pi
\end{equation*} 

\exercice
Développer la fonction $f$, de période $2\pi$, en série de Fourier.
\begin{equation*}
f(t)=
\begin{cases}
 1 & \text{si}\qquad t\in ]-\frac{\pi}{2},\frac{\pi}{2}] \\
 0 & \text{si}\qquad t\in ]\frac{\pi}{2},\frac{3\pi}{2}]
\end{cases}
\end{equation*} 

En déduire que :
\begin{equation*}
  1-\frac{1}{3}+\frac{1}{5}-\frac{1}{7}\dots=\frac{\pi}{4}
\end{equation*}

\exercice
Développer la fonction $f$, de période $2\pi$, en série de Fourier. Elle est définie sur $]-\pi,\pi]$ par :
\begin{equation*}
  f(t)=\frac{t^2}{4}
\end{equation*}
En déduire que :
\begin{eqnarray*}
  \sum_{n=1}^{\infty}\frac{1}{n^2}=\frac{\pi^2}{6} \\
 \sum_{n=1}^{\infty}(-1)^{n+1}\frac{1}{n^2}=\frac{\pi^2}{12} \\
 \sum_{n=1}^{\infty}\frac{1}{(2n-1)^2}=\frac{\pi^2}{8} \\
\end{eqnarray*}

\exercice
A partir des identités trigonométriques ci-dessous, déduire les séries de Fourier des fonctions $sin^3(t)$ et $cos^3(t)$.

\begin{eqnarray*}
  sin^3(t)=\frac{3}{4}sin(t)-\frac{1}{4}sin(3t) \\
 cos^3(t)=\frac{3}{4}cos(t)+\frac{1}{4}cos(3t) \\
\end{eqnarray*}

\exercice

Soit la fonction $2\pi$-périodique, définie par :
\begin{equation*}
f(t)=
\begin{cases}
 \frac{\pi}{8}t(\pi-t) & \text{si}\qquad t\in [0,\pi] \\
-\frac{\pi}{8}t(\pi+t) & \text{si}\qquad t\in [-\pi,0]
\end{cases}
\end{equation*} 
\begin{enumerate}
\item Représenter $f$ et la fonction $sin(t)$ sur la même courbe.
\item Calculer les séries de Fourier de $f$ et de $sin(t)$, puis les comparer.
\item Calculer $\int_0^{\pi}|f(t)-sin(t)|$dt
\item Calculer $g(t)=f'(t)$ et en déduire ses coefficients de Fourier.
\end{enumerate}


\exercice
Soit $f(t)=|t|$, définie pour $t\in]-\pi,\pi]$, de période $2\pi$. 
\begin{enumerate}
\item Calculer le développement en série de Fourier de $f$
\item Représenter graphiquement le spectre de $f$
\item En déduire que :
  \begin{equation*}
    \sum_{n\geq0}\frac{1}{(2n+1)^2}=\frac{\pi^2}{8} \qquad \text{et} \qquad \sum_{n\geq1}\frac{1}{n^2}=\frac{\pi^2}{6}
  \end{equation*}
\item En regardant leur tête et en utilisant l'outil adapté\footnote{Voilà un indice qui n'engage à rien sinon à réfléchir à deux fois avant de se lancer dans de grands calculs}, calculer :
 \begin{equation*}
    \sum_{n\geq0}\frac{1}{(2n+1)^4}=\frac{\pi^2}{96} \text{et} \sum_{n\geq1}\frac{1}{n^4}=\frac{\pi^2}{90}
  \end{equation*}
\end{enumerate}

\exercice
Soit la fonction $f$ de période $2\pi$, définie sur $[0,2\pi[$ par $f(t)=e^{iat}$, où $a$ est un paramètre réel.

  
\begin{itemize}
\item Calculer les coefficients de Fourier complexes de $f$
\item En utilisant l'égalité de Parseval montrer que :
\begin{equation}
\sum_{n=-\infty}^{\infty} \frac{1}{(a-n)^2}=\frac{\pi^2}{(sin(\pi a))^2}
\end{equation}
\end{itemize}




