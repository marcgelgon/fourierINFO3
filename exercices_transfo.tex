\exercice
%
Déterminer la transformée de Fourier de la fonction suivante (fonction
{ porte } ou { impulsion rectangulaire }) :
%
\begin{equation*}
p(t) =
\begin{cases}
1 & \mbox{si } t \in ] - \frac{a}{2},\frac a 2 ]\\
0 & \mbox{sinon}
\end{cases}
\end{equation*}

\exercice
%
Démontrer les propriétés suivantes de la transformation de Fourier :
\begin{enumerate}
\item dérivation (propriété~\ref{transfo-derivation}),
\item translation (propriété~\ref{transfo-translation}),
\item linéarité (propriété~\ref{transfo-linearite}).
\end{enumerate}

\exercice Déterminer la transformée de Fourier de la fonction suivante
(fonction { triangle } ou { impulsion triangulaire }) :
%
\begin{equation*}
f(t) =
\begin{cases}
-\frac 1 a  t + 1 & \mbox{si } t \in [ 0, a ]\\
\frac 1 a  t + 1 & \mbox{si } t \in [ -a, 0 [\\
0 & \mbox{sinon}
\end{cases}
\end{equation*}


\exercice En utilisant les propriétés de la transformée de Fourier,
déterminer la transformée de Fourier de la fonction triangle (donnée
ci-dessus) en utilisant le fait qu'on peut exprimer la dérivée de la
fonction triangle en fonction de la fonction porte.


\exercice \label{convol-porte} Déterminer le produit de convolution de la fonction porte
par elle-même.

\exercice 
Démontrer les propriétés suivantes du produit de convolution :
\begin{enumerate}

\item commutativité (propriété~\ref{convolution-commut}),

\item distributivité (propriété~\ref{convolution-distrib}),

\item dérivation (propriété~\ref{convolution-derivation}).

\end{enumerate}

\exercice Démontrer la propriété~\ref{convol_transfo1} du produit de convolution par rapport à la transformation de Fourier.

\exercice Déterminer la transformée de Fourier de la fonction triangle
en utilisant les propriétés du produit de convolution et le produit
de convolution de la fonction porte par elle-même.

\exercice Déterminer la transformée de Fourier de la fonction suivante définie sur $\mathbb R^2$ :
%
\begin{equation*}
f(x,y) =
\begin{cases}
 1  & \mbox{si } -1 \leq x \leq 1 \mbox{ et } -1 \leq y \leq 1\\
0 & \mbox{sinon}
\end{cases}
\end{equation*}


\exercice Déterminer la transformée de Fourier de la fonction suivante (gaussienne) :
$$f(t) = e^{-\pi t^2}$$