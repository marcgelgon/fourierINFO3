


\chapter{Séries de Fourier}

Joseph Fourier (1768 - 1830), Auxerrois de naissance, rapidement
orphelin, a initialement hésité entre les voies mathématique et
religieuse. Enseignant au collège de France et à Polytechnique, il a
été impliqué dans la révolution française, les expéditions
napoléoniennes et a été préfet de l'Isère, un peu contre son
gré. Cette fonction lui a toutefois laissé le temps de développer une
théorie concernant la décomposition de fonctions en séries
trigonométriques, dont ce cours est l'objet. En  1808, il l'a soumise
pour évaluation à Lagrange et Laplace, membres de l'Institut. L'article a alors été jugé moyennement clair et convaincant, ces jugements étant peut-être la preuve de l'originalité du travail, ou motivés par quelque concurrence pour un poste de prestige...

\section{De l'utilité d'étudier l'analyse de Fourier}

Dans la partie précédente, on a discuté de la manière dont on pouvait composer-décomposer (on parlera aussi de synthèse-analyse) une fonction  comme une combinaison linéaire de fonctions simples. De façon générale, la décomposition de données et signaux comme des combinaisons de fonctions simples est une opération centrale en informatique, pour le traitement efficace de ces données et signaux. Ces décompositions fournissent de nouvelles représentations des signaux qui faciliteront la compression (une représentation des données qui est économe mais qui parvient à néanmoins représenter assez fidèlement les données initialement, notamment les signaux audio, image et vidéo), l'interprétation (la compréhension de phénomènes portés dans les données et les signaux), le filtrage (c.a.d. la sélection de sous-parties intéressantes des données). Ces décompositions (parfois qualifiées de parcimonieuses) sont aussi en lien étroit avec l'apprentissage (machine learning) dont, par exemple, les représentations internes des réseaux de neurones. Des sujets proches apparaîtront aussi dans des enseignements d'analyse statistique des données (analyse en composantes principales).

Dans ce cadre général, l'analyse de Fourier est un cas particulier où les fonctions de base sont fixées et sont les fonctions trigonométriques. Dans d'autres sous-domaines, on utilise d'autres bases, ou bien on cherche des bases adaptatives et spécifiquement optimales pour les données à traiter. L'analyse de Fourier est, en particulier, très utilisée pour le traitement des images et sons et les questions de transmission de données, notamment parce qu'elle décompose les signaux selon l'axe des fréquences et que beaucoup de phénomènes ont des propriétés physiques ou perceptuelles qui varient selon la fréquence.


Par ailleurs, l'analyse de Fourier est, à l'occasion, un outil de calcul simplifiant ou permettant certains calculs d'intégrales.


\section{Définition des séries de Fourier et calcul de leur coefficients}

Soit
%\footnote{Plus généralement, il faudrait considérer les fonctions
%  de $\R$ vers $\C$ mais on s'en dispensera dans ce chapitre.}
 $f$ fonction périodique de période $2\pi$, d'une variable réelle, vérifiant certaines propriétés de continuité, qui seront précisées en section~\ref{sec_convergence}.

%On peut ré-écrire $e^{inx}=cos(nx)\; +\; i\;sin(nx)$.

% \footnote{ce point sera précisé à la section XXX. En
%  pratique, ces conditions sont remplies dans plupart des fonctions
% rencontrées.}

\begin{boxedminipage}{14.5cm}
L'objectif des séries de Fourier est de considérer $f$ comme une
combinaison linéaire (généralement infinie) de fonctions trigonométriques élémentaires :
\begin{equation}
f(x)=a_0+\sum_{n=1}^{\infty}[a_n cos(nx) + b_n sin(nx)]
\label{defseries}
\end{equation}
où $a_0$, les $a_n$ et les $b_n$ sont des réels.
\end{boxedminipage}
%Cette expression montre qu'il s'agit de décomposer $f(x)$ comme une
 % combinaisonlinéaire infinie de fonctions élémentaires $cos(nx)$ et
 % $sin(nx)$, c.a.d. une combinaison linéaire de $ cos(x), sin(x),$$cos(2x), sin(2x),
 % ....$

Pour comprendre l'objectif général, on gagne à regarder tout de suite la figure \ref{fig:exemple_approx_carre} qui illustre l'expression d'une fonction $f$ comme une somme de fonctions trigonométriques.

Il est bon de mettre un coup de projecteur sur une caractéristique
essentielle de la définition (\ref{defseries}) : les fonctions
${cos(x), sin(x), cos(2x), sin(2x),...}$ forment une famille orthogonale, mais pas encore orthonormée.


Modifions légèrement la définition du produit scalaire entre deux fonctions $f$ et $g$ qui avait été présentée en section \ref{sec:prodscal}, comme suit :
\begin{equation}
\langle f, g \rangle  = \frac{1}{\pi}\int_0^{2\pi} f(x)\; g(x)\; dx
\end{equation}

En section \ref{sec:prodscal}, il avait été précisé qu'il n'y a pas de définition unique du produit scalaire, mais il faut et il suffit de vérifier les propriétés de forme bilinéaire semi-définie positive, qui sont toujours respectée par cette nouvelle définition. 

Il en résulte que la définition de la norme d'une fonction est, elle aussi, adaptée, comme suit :
\begin{equation}
\|f\|= \sqrt{\langle f, f \rangle}= \sqrt{ \frac{1}{\pi}\int_0^{2\pi} f(t)^2dt}
\end{equation} 

Pour s'embêter avec ces nouvelles définitions ? Parce qu'ainsi :
\begin{itemize}
  \item La famille $\{sin(x),cos(x),\dots,sin(nx),cos(nx),\dots...,\}$ est une famille orthonormée et pas seulement orthogonale.
  \item On peut identifier les \emph{coefficients de Fourier} $a_0$, $a_n$ et
 $b_n$, coefficients de la combinaison linéaire, comme suit :
 $\forall u\in \R$ :
\begin{eqnarray}
& a_0 & =\frac{1}{2\pi}\int_{u}^{u+2\pi}f(x)dx \label{a0}\\
\text{pour} ~ n\in \N^* & a_n & = \langle f, cos(nx) \rangle = \frac{1}{\pi}\int_{u}^{u+2\pi}f(x)~ cos(nx)~ dx  \label{an}\\
\text{pour} ~ n\in \N^* & b_n & = \langle f, sin(nx) \rangle= \frac{1}{\pi}\int_{u}^{u+2\pi}f(x)~ sin(nx) dx  \label{bn}
\end{eqnarray}
  \end{itemize}


On trouve ici la justification du chapitre 1 du document : chaque coefficient $a_n$ et $b_n$ est le résultat du produit
  scalaire usuel entre la fonction $f$ et une des fonctions de la base des
  séries de Fourier. De manière analogue au produit scalaire dans
  $\R^n$, on peut voir cette opération comme une projection de $f$
  sur chaque fonction de base, permettant d'identifier les
  coefficients de sa décomposition comme une combinaison linéaire de
  fonctions trigonométriques simples. En prenant le point de vue
  constructif, on peut aussi dire qu'on \emph{synthétise}  $f$ comme une combinaison linéaire de fonctions de base.

Remarques  :

\begin{enumerate}
\item Les trois intégrales \ref{a0} à \ref{bn}  couvrent la période de $f$,
  mais comme l'indique ``$\forall u\in \R$'', le début et la fin de
  l'intervalle d'intégration sont sans importance, du moment que
  l'intervalle est une période de $f$. On peut  donc choisir une
  valeur de $u$ qui rend le calcul le plus simple possible (par exemple $u=0$ ou $u=-\pi$, dans le cas d'une période $2\pi$).
\item $a_0$ peut être interprété comme la \emph{valeur
  moyenne\footnote{Attention : il existe une autre définition de la
  série de Fourier, où ce coefficient $a_0$ est affecté d'un
  coefficient multiplicateur $1/2$. Ca n'est guère un problème, mais
  les expressions de calculs des coefficients doivent être adaptées à
  ce choix.}} de $f$ sur la
  période. Il est essentiel de le voir par sa définition, et de garder
  cette idée à l'esprit, notamment pour vérifier, que la
  valeur calculée est raisonnable. Cette interprétation montre aussi que les autres
  coefficients $a_n$ et $b_n$ synthétisent les \emph{variations} de
  $f$ autour de sa moyenne $a_0$.
\end{enumerate}

\section{Reconstruction approximative} 

Contrairement à la décomposition de vecteurs dans $\R^n$ où on peut reconstruire de manière exacte n'importe quel vecteur par une somme \emph{finie} parce qu'on est dans un espace de dimension finie, on a généralement besoin d'une somme infinie et qu'on fait cette décomposition dans un espace de fonctions de dimension infinie. La base $\{1,sin(x),cos(x),\dots,sin(nx),cos(nx),\dots...\}$ est donc composée d'une infinité de fonctions.

En restreignant la série de Fourier d'une fonction $f$ à une somme
\emph{finie}, formée par les $N$ premières composantes de la série, on obtient une
\emph{approximation} de $f$ (dite d'ordre $N$). L'erreur d'approximation peut s'exprimer comme suit :
\begin{equation}
err(x)=\underbrace{  f(x)- \Big( \underbrace{a_0+\sum_{n=1}^{N}[a_n cos(nx) + b_n sin(nx)]\Big)}_{\text{reconstruction approximative}}}_{\text{erreur d'approximation}}
\label{erreur_approx}
\end{equation}

Une manière usuelle de qualifier la qualité d'une approximation est
l'erreur quadratique $\int_{-\pi}^{\pi}err(x)^2 dx$.
Vous entendrez aussi souvent l'expression ``moindres carrés'' pour
dénommer ce critère.

Une bonne propriété de l'approximation par une série finie de
Fourier est qu'étant donné le choix de la famille de fonctions
trigonométriques, les coefficients obtenus par les expressions
\ref{a0} à \ref{bn}, pour le $N$ qu'on s'est fixé, aboutissent une
approximation optimale au sens que l'erreur quadratique est minimale.
Pour des fonctions $f$ couvrant une grande diversité de signaux périodiques rencontrés couramment, cela conduit à produit de bonnes approximations de $f$ avec des approximations ne nécessitant qu'un faible nombre de termes de la série de Fourier.

La figure~\ref{fig:exemple_approx_carre} fournit un exemple décomposition en séries de Fourier pour la fonction $f$, périodique de période $2\pi$, qui vaut 1 de 0 à $\pi$ et -1 de $\pi$ à $2\pi$. 

\begin{figure}
\begin{center}
\begin{tikzpicture}[background rectangle/.style={fill=olive!5}, show background rectangle]
\begin{groupplot}[group style={group size=2 by 2, horizontal sep=2cm, vertical sep=2cm},scale=0.8,samples=150]
\nextgroupplot[title={Fonction $f$ à décomposer},ymin=-1.5,ymax=1.5]
\addplot[red,ultra thick,domain=0:3.14]{1};
\addplot[red,ultra thick,domain=3.14:6.28]{-1};
\addplot[red,ultra thick,domain=6.28:6.28+3.14]{1};
\addplot[red,ultra thick,domain=6.28+3.14:6.28+2*3.14]{-1};
\addplot[red,ultra thick,domain=6.28+2*3.14:6.28+3*3.14]{+1};
\nextgroupplot[title={N=1},ymin=-1.5,ymax=1.5]
  \addplot[blue, ultra thick,domain=0:3.14*5] {4/3.14*sin(360/6.28*x)};
\addplot[red,ultra thick,domain=0:3.14]{1};
\addplot[red,ultra thick,domain=3.14:6.28]{-1};
\addplot[red,ultra thick,domain=6.28:6.28+3.14]{1};
\addplot[red,ultra thick,domain=6.28+3.14:6.28+2*3.14]{-1};
\addplot[red,ultra thick,domain=6.28+2*3.14:6.28+3*3.14]{+1};
\nextgroupplot[title={N=3},ymin=-1.5,ymax=1.5]
  \addplot[blue, ultra thick,domain=0:3.14*5] {4/3.14*sin(360/6.28*x)+0.33*4/3.14*sin(3*360/6.28*x)};
 \addplot[red,ultra thick,domain=0:3.14]{1};
\addplot[red,ultra thick,domain=3.14:6.28]{-1};
\addplot[red,ultra thick,domain=6.28:6.28+3.14]{1};
\addplot[red,ultra thick,domain=6.28+3.14:6.28+2*3.14]{-1};
\addplot[red,ultra thick,domain=6.28+2*3.14:6.28+3*3.14]{+1};
\nextgroupplot[title={N=5},ymin=-1.5,ymax=1.5]
  \addplot[blue,ultra thick,domain=0:3.14*5] {4/3.14*sin(360/6.28*x)+0.33*4/3.14*sin(3*360/6.28*x)+1/5*4/3.14*sin(5*360/6.28*x)};
\addplot[red,ultra thick,domain=0:3.14]{1};
\addplot[red,ultra thick,domain=3.14:6.28]{-1};
\addplot[red,ultra thick,domain=6.28:6.28+3.14]{1};
\addplot[red,ultra thick,domain=6.28+3.14:6.28+2*3.14]{-1};
\addplot[red,ultra thick,domain=6.28+2*3.14:6.28+3*3.14]{+1};
\end{groupplot}

\node[align=center,font=\bfseries, yshift=2em] (title) 
    at (current bounding box.north)
    {Décomposition - synthèse };

\end{tikzpicture}
\end{center}
\caption{Approximations d'une fonction "créneau", 3 approximations de $f$, notées $f_1,f_3,f_5$, de plus en plus précises avec $N$ croissant. On poursuivant la somme vers une valeur de $N$ suffisamment élevée, on pourrait atteindre une qualité d'approximation arbitraire.}
\label{fig:exemple_approx_carre}
\end{figure}

Le calcul donne : $a_0=\frac{4}{\pi}$, $a_n=0$ pour tout $n>0$, $b_n=0$ si $n>0$ et pair, $b_n=\frac{4}{n\pi}$

L'expression de $f$ en série de Fourier est donc :
\begin{equation}
f(x)=\sum_{p=1}^\infty \frac{4}{(2p-1)\pi}sin((2p-1)x)
\end{equation}
\ref{defseries}.

\begin{eqnarray}
\text{Pour $N=1$ , on trouve } f_1(x) & =\frac{4}{\pi}sin(x) & \\
\text{Pour $N=3$ , on trouve } f_3(x) & =\frac{4}{\pi}\Big(sin(x) & +\frac{1}{3}sin(3x) \Big) \\
\text{Pour $N=5$  , on trouve } f_5(x) & =\frac{4}{\pi}\Big(sin(x)& +\frac{1}{3}sin(3x)+\frac{1}{5}sin(5x)\Big) 
\end{eqnarray}


Remarquons que :
\begin{itemize}
\item la fonction ``créneau'' n'est pas continue mais elle a l'amabilité
  de vérifier les conditions de Dirichlet. La reconstuction autour des points de discontinuité présente les caractéristiques suivantes :
\begin{itemize}
\item l'approximation obtenue à l'abscisse de la discontinuité est la
  moyenne de la limite à droite et à gauche, conformément à ce qui est
  raconté à la section ~\ref{sec_convergence},
\item les discontinuités de la fonction ont du mal à être
  approximées correctement : on appelle ``phénomène de Gibbs'' ces oscillations aux
  alentours du point de discontinuité\footnote{On peut s'en
  débarrasser avec une ruse nommée ``facteur de Lanczos'', qu'on n'approfondit pas dans ce cours.}
\end{itemize}
\end{itemize} 


\section{Conditions de convergence de la série de Fourier}
\label{sec_convergence}
Voici des conditions \emph{suffisantes} pour que la décomposition en série de
Fourier s'applique correctement. Si :
\begin{itemize}
\item $f$ est continue par morceaux, i.e. elle comporte un nombre fini
  de discontinuités et, à chacun de ces points de discontinuité, il
  existe des limites \emph{finies} à droite et à gauche.
\item $f$ comporte un nombre fini de maxima et de minima.
%\item et sa dérivée $f'$ possède les mêmes propriétés
\end{itemize}
alors la série de Fourier dont les coefficients sont définis par les
expressions \ref{a0} à \ref{bn} converge vers :
\begin{itemize}
\item $f(x)$ aux points où $f$ est continue 
\item vers $[f(x^+)+f(x^-)]/2$ aux points où $f$ est discontinue.
\end{itemize}

Ces conditions \emph{suffisantes (pas nécessaires)} sont dites
\emph{conditions de Dirichlet\footnote{Peter Gustav Lejeune-Dirichlet
  ( 1805 - 1859 )}}. En pratique, sur les données et
signaux que l'on manipule dans le métier d'ingénieur, ces conditions
sont presque toujours remplies, et ces questions de convergences ne
posent pas grand problème.


\section{Cas d'une fonction de période quelconque}
Tous les principes des séries de Fourier s'appliquent aussi à des périodes
autres que $2\pi$, il faut simplement ajuster
toutes les expressions. Dans le cas d'une fonction $f$ de période
$T$\footnote{donc de \emph{pulsation fondamentale}
  $\omega=\frac{2\pi}{T}$}, en faisant le changement de variable
$\frac{t}{T}=\frac{x}{2\pi}$, on trouve l'expression de la décomposition de $f$ :

\begin{eqnarray}
f(x) & = A_0+\sum_{n=1}^{\infty}[A_n cos(n\frac{2\pi}{T}x) + B_n sin(n\frac{2\pi}{T}x)] \\
     & = A_0+\sum_{n=1}^{\infty}[A_n cos(n\omega x) + B_n sin(n\omega x)] 
\end{eqnarray}

\begin{eqnarray}
& A_0 & =\frac{1}{T}\int_{u}^{u+T}f(x)dx \label{a0T}\\
n\in \N^* & A_n & =\frac{2}{T}\int_{u}^{u+T}f(x)\ cos(n\omega~x) dx  \label{anT}\\
n\in \N^* & B_n & =\frac{2}{T}\int_{u}^{u+T}f(x)\ sin(n\omega~x) dx  \label{bnT}
\end{eqnarray}

où à $u$ s'applique la remarque déjà faite pour le cas d'une période quelconque.

\section{Cas où le calcul des coefficients $a_n$ et $b_n$ se simplifie}

Certaines éventuelles propriétés de $f$ (supposée de période $2\pi$) 
permettent des réécritures des expressions ~\ref{a0},~\ref{an} et
~\ref{bn}, qui permettent souvent de simplifier le calcul de ces coefficients :
\subsection{Si $f$ est paire}
\begin{eqnarray}
& a_0=&\frac{1}{\pi}\int_{0}^{\pi}f(x)dx \label{a0paire}\\
n\in \N^* & a_n=&\frac{2}{\pi}\int_{0}^{\pi}f(x)\ cos(nx) dx  \label{anpaire}\\
n\in \N^* & b_n=&0  \label{bnpaire}
\end{eqnarray}
Autrement dit, on peut synthétiser une fonction paire à partir des
seules composantes en cosinus (plus éventuellement une constante $a_0$).
\subsection{Si $f$ est impaire}
\begin{eqnarray}
& a_0=&0\label{aimpaire0}\\
n\in \N^* & a_n=&0  \label{animpaire}\\
n\in \N^* & b_n=& \frac{2}{\pi}\int_{0}^{\pi}f(x)\ sin(nx) dx \label{bnimpaire}
\end{eqnarray}
%\subsection{Coefficients d'une dérivée}
%\subsection{Effet de la translation sur les coefficients de Fourier}

Bien entendu les expressions ~\ref{a0},~\ref{an} et ~\ref{bn} sont
toujours vraies et utilisables

\section{La série de Fourier sous la forme (amplitudes, déphasages)}

On trouvera ici un élément de réponse à la question suivante : "pourquoi faut-il à la fois des cosinus et des sinus dans la base ?"

La figure~\ref{fig:effet_translation} montre trois fonctions créneau périodiques (en fait, elles sont toutes égales à une translation près le long de l'axe des abscisses). On appellera parfois cette translation \emph{déphasage}, dans la mesure où ces fonctions sont périodiques; si l'axe des abscisses était l'axe du temps, ça correspondrait à des signaux légèrement en avance ou en retard les uns par rapport aux autres. La première fonction est impaire et sa série ne nécessite que des termes en sinus. La seconde fonction est paire et sa série ne nécessite que des termes en cosinus. La troisième fonction, qui est le cas général, n'est ni paire ni impaire et sa série contient des termes non nuls en sinus et cosinus. Les approximations de ces fonctions à l'ordre 1 sont :

\begin{itemize}
  \item $f(x) ~ \approx ~\frac{4}{\pi}sin(x)$ c'est à dire $b_1=\frac{4}{\pi}$
  \item $g(x) ~ \approx ~ \frac{4}{\pi}cos(x)$ c'est à dire $a_1=\frac{4}{\pi}$
  \item $h(x) ~ \approx ~ \frac{4}{\pi}\Big(cos(x)cos(\frac{\pi}{4})+sin(x)sin(\frac{\pi}{4})$\Big)  c'est à dire $a_1=\frac{4}{\pi}.\frac{\sqrt 2}{2}$ et $b_1=\frac{4}{\pi}.\frac{\sqrt 2 }{2}$
\end{itemize}

En utilisant la propriété $cos(\theta_1+\theta_2)=cos(\theta_1)cos(\theta_2)-sin(\theta_1)sin(\theta_2)$, la définition des séries de Fourier :
\begin{equation}
f(x)=a_0+\sum_{n=1}^{\infty}~\Big(a_n cos(nx) + b_n sin(nx)\Big)
\end{equation}
peut se ré-écrire sous la forme :
\begin{equation}
f(x)=a_0+\sum_{n=1}^{\infty}[\alpha_n cos(nx+\beta_n)]
\end{equation}
où $\alpha_n$ est une amplitude et $\beta_n$ est un déphasage.

Cette nouvelle manière d'écrire les séries de Fourier permet de les voir comme une somme de fonctions cosinus munies d'amplitude et de déphasages, que l'expression sous forme de somme de sinus et cosinus de même fréquence rend moins visible. 
%En particulier, cette nouvelle écriture permet de voir la troisième fonction de la figure~\ref{fig:effet_translation} comme

Un petit calcul montre que \quad $\alpha_n=\sqrt{a_n^2+b_n^2}$ \quad et \quad $\beta_n=- \arctan \frac{a_n}{b_n}$ (si $b_n \neq 0$). 

On pourrait montrer une transformation semblable vers une somme de sinus plutôt que cosinus.

\begin{figure}
\begin{center}
\begin{tikzpicture}[background rectangle/.style={fill=olive!5}, show background rectangle]
\begin{groupplot}[group style={group size=3 by 2, horizontal sep=1cm, vertical sep=2cm},scale=0.7,samples=150,grid=major]
\nextgroupplot[title={$f(x)$},ymin=-1.5,ymax=1.5]
\addplot[red,ultra thick,domain=-6.26:-3.14]{1};
\addplot[red,ultra thick,domain=-3.14:0]{-1};
\addplot[red,ultra thick,domain=0:3.14]{1};
\addplot[red,ultra thick,domain=3.14:6.28]{-1};
\nextgroupplot[title={$g(x)$},ymin=-1.5,ymax=1.5]
\addplot[red,ultra thick,domain=-6.28:-3.14-1.57]{1};
\addplot[red,ultra thick,domain=-3.14-1.57:-1.57]{-1};
\addplot[red,ultra thick,domain=-1.57:1.57]{1};
\addplot[red,ultra thick,domain=1.57:1.57+3.14]{-1};
\addplot[red,ultra thick,domain=1.57+3.14:6.28]{-1};
\nextgroupplot[title={$h(x)$},ymin=-1.5,ymax=1.5]
\addplot[red,ultra thick,domain=-6.28+0.78:-3.14-1.57+0.78]{1};
\addplot[red,ultra thick,domain=-3.14-1.57+0.78:-1.57+0.78]{-1};
\addplot[red,ultra thick,domain=-1.57+0.78:1.57+0.78]{1};
\addplot[red,ultra thick,domain=1.57+0.78:1.57+3.14+0.78]{-1};
\addplot[red,ultra thick,domain=1.57+3.14+0.78:6.28+0.78]{-1};



\nextgroupplot[title={Approx. par sin suffit},ymin=-1.5,ymax=1.5]
  \addplot[blue, ultra thick,domain=-6.28:6.28] {4/3.14*sin(360/6.28*x)};
\addplot[red,ultra thick,domain=-6.26:-3.14]{1};
\addplot[red,ultra thick,domain=-3.14:0]{-1};
\addplot[red,ultra thick,domain=0:3.14]{1};
\addplot[red,ultra thick,domain=3.14:6.28]{-1};
\nextgroupplot[title={Approx. par cos suffit},ymin=-1.5,ymax=1.5]
  \addplot[blue, ultra thick,domain=-6.28:6.28] {4/3.14*cos(360/6.28*x)};
\addplot[red,ultra thick,domain=-3.14-1.57:-1.57]{-1};
\addplot[red,ultra thick,domain=-1.57:1.57]{1};
\addplot[red,ultra thick,domain=1.57:1.57+3.14]{-1};
\nextgroupplot[title={Approx. par sin et cos},ymin=-1.5,ymax=1.5]
 \addplot[blue, ultra thick,domain=-6.28:6.28] {4/3.14*cos((360)/6.28*x)*0.70+4/3.14*sin((360)/6.28*x)*0.70};
\addplot[red,ultra thick,domain=-6.28+0.78:-3.14-1.57+0.78]{1};
\addplot[red,ultra thick,domain=-3.14-1.57+0.78:-1.57+0.78]{-1};
\addplot[red,ultra thick,domain=-1.57+0.78:1.57+0.78]{1};
\addplot[red,ultra thick,domain=1.57+0.78:1.57+3.14+0.78]{-1};
\addplot[red,ultra thick,domain=1.57+3.14+0.78:6.28+0.78]{-1};

\end{groupplot}

\node[align=center,font=\bfseries, yshift=2em] (title) 
    at (current bounding box.north)
    {Effet de la translation horizontale};

\end{tikzpicture}
\end{center}
\caption{La figure montre trois variantes de la fonction créneau, notée ici $f$ correspondant à la même fonction, translatée horizontalement avec, à chaque fois, l'approximation de Fourier à l'ordre 1 (les coefficients non cités étant nuls).}
\label{fig:effet_translation}
\end{figure}


\section{Notation complexe}

Si on voulait être synthétique, on pourrait présenter la série de
Fourier comme une somme de fonctions périodiques de la forme :
\begin{eqnarray}
\R & \rightarrow \C \\
x & \mapsto e^{inx}
\end{eqnarray}
Supposons $f$ de période $2\pi$. 
La décomposition linéaire s'écrit :
\begin{equation}
f(x)=\sum_{-\infty}^{\infty}\; c_n e^{inx}
\end{equation}

Les coefficients de cette décomposition peuvent se déterminer par :
\begin{equation}
c_n=\frac{1}{2\pi}\int_{u}^{u+2\pi} f(x)e^{-inx}dx
\label{version_complexe}
\end{equation}

et ces coefficients complexes vérifient :
\begin{eqnarray}
c_0=&a_0 \\ 
c_n=&\frac{1}{2}(a_n-ib_n) \\
c_{-n}=& \overline{c_n} \text{~~~(complexe conjugué)}
\end{eqnarray}

Cette autre manière d'aboutir à la série de Fourier :
\begin{itemize}
\item mène parfois à un calcul plus facile (par exemple, si $f(x)$ est une fonction exponentielle)
\item est plus proche de la forme de la transformée de Fourier, et permettra de voir que la série de Fourier est un cas particulier de la transformée de Fourier.
\end{itemize}


\section{Coefficients de Fourier de transformées d'une fonction}
L'application d'une dérivation, d'une translation etc...  sur une
fonction $f$ a un effet particulier sur ses coefficients de Fourier, 
Autrement dit, on peut déduire directement les coefficients de la fonction transformée
des coefficients de la fonction originale.

\medskip Il est utile de savoir démontrer ces propriétés, car les techniques utilisées sont d'une utilisation très communs (intégration par partie, changement de variable,...).

\subsection{Coefficients de Fourier d'une dérivée}
Soit $f$ une fonction de période $2\pi$ vérifiant les conditions de Dirichlet.
Alors : 
\begin{eqnarray}
& a_0(f')&=0 \label{eqderivee}\\
\text{si~} n\in \N^* & a_n(f')&=n~b_n(f) \\
\text{si~} n\in \N^* & b_n(f')&=-n~a_n(f) \\
\text{si~} n\in \N^* & c_n(f')&=i~n~c_n(f)
\end{eqnarray}
Plus généralement, si $f$ est de période $T$, c'est-à-dire de
pulsation $\omega=\frac{2\pi}{T}$ :
\begin{eqnarray}
& a_0(f')&=0 \\
\text{si~}n\in \N^* & a_n(f')&=n~\omega~b_n(f) \\
\text{si~}n\in \N^* & b_n(f')&=-n~\omega~a_n(f) \\
\text{si~}n\in \N^* & c_n(f')&=i~n~\omega~c_n(f)
\end{eqnarray}

\subsection{Coefficients de Fourier d'une translatée}

Dans le cas des coefficients complexes, le résultat est facile à
montrer et particulièrement compact :

Si $g(x)=f(x-y)$, alors $c_n(g)=e^{-iny}c_n(f)$ et vice-versa.

Si on considère qu'il s'agit d'une translation d'un signal dans le temps, il est rassurant de vérifier que le spectre $\|c_n\|$, c.a.d. la répartition de l'énergie du signal selon les fréquences, est inchangé par translation. 

%Voir plutôt le cas de la transformation de Fourier.

\section{Linéarité de l'application}

Si $f$ et $g$ sont deux fonctions vérifiant les conditions de
Dirichlet, toutes deux de période $T$, alors :
\begin{eqnarray}
a_n(\lambda f+\mu g)=\lambda a_n(f)  + \mu a_n (g) \\
b_n(\lambda f+\mu g)=\lambda b_n(f)  + \mu b_n (g) \\
c_n(\lambda f+\mu g)=\lambda c_n(f)  + \mu c_n (g)
\end{eqnarray}

Cette propriété découle immédiatement de la linéarité de l'intégrale. On a tout intérêt à l'utiliser pour simplifier le calcul de coefficients en série de Fourier,  si la fonction à traiter se décompose comme la somme (plus généralement, combinaison linéaire) de plusieurs fonctions élémentaires dont les coefficients se calculent plus simplement. Il suffit alors de combiner facilement les coefficients obtenus sur les fonctions élémentaires.

\section{Egalité de Parseval}

Une propriété importante des séries de Fourier est le théorème de
Parseval. Soit une fonction $f$, périodique de période $T$. Dans les cas des coefficients complexes et réels, l'égalité de Parseval s'écrit :
\begin{equation}
a_0^2+\frac{1}{2}\sum_{n=1}^\infty (a_n^2+b_n^2)=\quad \sum_{n=-\infty}^\infty | c_n|^2 \quad=\quad \frac{1}{T}\int_{periode~T}|f(x)|^2 dx
\end{equation}

Une interprétation de cette propriété est que l'énergie du signal sur la période est la même, qu'elle soit mesurée dans le domaine "direct" $x$ (temporel, souvent), ou dans le domaine de Fourier. 

Un point pratique qui découle de cette expression est que les sommes $a_0^2 + \frac{1}{2}\sum_{n=1}^\infty (a_n^2+b_n^2)=\quad\sum_{n=-\infty}^\infty | c_n|^2 $ sont convergentes, ce qui implique que les coefficients de Fourier doivent décroître suffisamment vite en fonction de $n$. Si vos calculs aboutissent à des coefficients qui décroissent trop lentement avec $n$, voire augmentent, c'est qu'il faut faire la chasse aux erreurs...

Cette expression permet aussi de calculer la valeur de certaines séries : les exercices 13 et 14 l'illustrent.

% ci-dessous OSEF on ferait mieux d'évoquer Pythagore éventuellement
%L'égalité de Parseval permet aussi la démonstration d'une propriété bien connue : ``soit, dans $\R^2$, l'ensemble courbes fermées. La courbe de longueur minimale dont l'intérieur définit un domaine d'aire fixée est un cercle.''



